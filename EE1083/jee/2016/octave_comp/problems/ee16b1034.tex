The point of intersection of the two lines is one vertex of the rhombus.  This point is obtained by soving the following matrix equation 
%
\begin{equation}
\begin{pmatrix}
1 & -1
\\
7 & -1
\end{pmatrix}
\mbf{x} = 
\begin{pmatrix}
-1
\\
5
\end{pmatrix}
\end{equation}
%
using the octave code to obtain the point $P$ $\brak{1,2}$.  Since diagonals of a rhombus bisect each other and the point of intersection $O$ is given as $\brak{-1,-2}$ the coordinates of the opposite vertex $R$ are given by 
%
\begin{equation}
\mathbf{x} + 
\begin{pmatrix}
1 
\\
2
\end{pmatrix}
= 
2
\begin{pmatrix}
-1 
\\
-2
\end{pmatrix} \Rightarrow \mbf{x} = 
\begin{pmatrix}
-3
\\
-6
\end{pmatrix}
\end{equation}
%
Since the sides of a rhombus are equal, if the unknown vertex $Q$ has coordinates $\brak{x,y}$,
\begin{align}
PQ = QR \Rightarrow \brak{x-1}^2 + \brak{y-2}^2 &= \brak{x+3}^2  + \brak{y+6}^2 \nonumber \\
\Rightarrow x + 2 y &= -5 
\end{align}
Note that the above locus is actually the diagonal $QS$.  Letting $PQ$ be
\begin{equation}
x-y+1 = 0,
\end{equation}
%
$Q$ is obtained from the following equation
%
%
\begin{equation}
\begin{pmatrix}
1 & 2
\\
1 & -1
\end{pmatrix}
\mbf{x} = 
\begin{pmatrix}
-5
\\
-1
\end{pmatrix}
\end{equation}
%
as $\brak{-\frac{7}{3},-\frac{4}{3}}$.
%
Similarly, letting $PS$ to be
%
\begin{equation}
7x-y-5=0,
\end{equation}
%
$S$ is obtained from the equation
%
\begin{equation}
\begin{pmatrix}
1 & 2
\\
7 & -1
\end{pmatrix}
\mbf{x} = 
\begin{pmatrix}
-5
\\
5
\end{pmatrix}
\end{equation}
%
as $\brak{\frac{1}{3},-\frac{8}{3}}$.


