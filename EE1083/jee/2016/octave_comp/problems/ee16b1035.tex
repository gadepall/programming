The given circle can be expressed in standard form as
%
\begin{equation}
\brak{x-4}^2 + \brak{y-4}^2 = 6^2
\end{equation}
%
i.e., the cricle has centre at $\brak{4,4}$ and radius 6.
Let $\brak{h,k}$ be the centre of a circle that touches the given circle. Since this  circle touches the $x$-axis, its radius is $k$. This circle can touch the given circle internally or externally.
\begin{enumerate}
\item {\em External:}  In this case, sum of radius of two circles is equal to distance between them.  Hence, 
%
\begin{align}
|k|+6&=\sqrt{(h-4)^{2}+(k-4)^{2}}
\\
\Rightarrow k^2 + 12\abs{k} + 36 &= \brak{h-4}^2 + k^2 -8k + 16 \\
\Rightarrow  12\abs{k} +8k &= \brak{h-4}^2 - 20\\
\Rightarrow  k &= 
\begin{cases}
\frac{\brak{h-4}^2 - 20}{20} & k  > 0
\\
-\frac{\brak{h-4}^2 - 20}{4} & k < 0
\end{cases}
\end{align}
%
\item {\em Internal:} Modulus of difference of radius of two circles is equal to distance between them.  hence
%
\begin{align}
\abs{|k|-6}&=\sqrt{(h-4)^{2}+(k-4)^{2}}
\\
\Rightarrow k^2 - 12\abs{k} + 36 &= \brak{h-4}^2 + k^2 -8k + 16 \\
\Rightarrow  -12\abs{k} +8k &= \brak{h-4}^2 - 20\\
\Rightarrow  k &= 
\begin{cases}
\frac{\brak{h-4}^2 - 20}{20} & k  < 0
\\
-\frac{\brak{h-4}^2 - 20}{4} & k > 0
\end{cases}
\end{align}
%
\end{enumerate}
%
Both the above cases can be combined to obtain the locus as the curves
%
\begin{align}
y &= \frac{\brak{x-4}^2 }{20} - 1
\\
y &= 5 - \frac{\brak{x-4}^2 }{4} 
\end{align}
%
